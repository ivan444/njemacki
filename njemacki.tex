\documentclass[12pt,german]{article}
\usepackage{amsmath}
\usepackage{amssymb}
\usepackage[croatian,german]{babel}
\usepackage{booktabs}
\usepackage[bf]{caption}
\usepackage{enumitem}
\usepackage{float}
\usepackage[T1]{fontenc}
\usepackage[a4paper, left=0.5cm, right=0.5cm, bottom=2cm, top=0.5cm]{geometry}
\usepackage{graphicx}
\usepackage[utf8]{inputenc}
\usepackage{lmodern}
\usepackage{multicol}
\usepackage{multirow}
\usepackage{rotating}
\usepackage{titlesec}
\usepackage[flushleft]{threeparttable}
\usepackage[
    left = \glqq,%
    right = \grqq%
]{dirtytalk}

\captionsetup{labelsep=space,justification=justified,singlelinecheck=off}

% Custom section: "1 | Title"
\titleformat{\section}[block]
{\normalfont\Large\bfseries}
{\thesection\hskip 9pt$|$\hskip 9pt}
{0pt}
{}

\newcommand{\nastavak}[1]{\emph{-#1}}
\newcommand{\prefiks}[1]{\emph{#1-}}
\newcommand{\clan}[1]{\emph{#1}}
\newcommand{\primjer}[2]{\say{#1 $\longmapsto$ #2}}
\newcommand{\primjerNul}[1]{\say{#1}}
\newcommand{\primjerTri}[3]{\say{#1 $\longmapsto$ #2 $\longmapsto$ #3}}
\newcommand{\prijevod}[2]{\item #1 (\emph{#2})}

\author{Ivan Krišto}

\begin{document}
\title{Sažetak njemačke gramatike}
\date{}
\maketitle

% 11.
% Primjeri nisu svi ispravni. Ne da mi se pisati, pitaj me u živo :P

\begin{multicols}{2}
\section{Rodovi}

\begin{table}[H]
\caption{Određeni članovi.}
\begin{tabular}{lllll}
\toprule
Rod & Nominativ & Genitiv & Dativ & Akuzativ \\
\midrule
Muški & der & des & dem & den \\
Ženski & die & der & der & die \\
Srednji & das & des & dem & das \\
Množina & die & der & den & die \\
\bottomrule
\end{tabular}
\end{table}

\begin{table}[H]
\caption{Neodređeni članovi.}
\begin{tabular}{lllll}
\toprule
Rod & Nominativ & Genitiv & Dativ & Akuzativ \\
\midrule
Muški & ein & eines & einem & einen \\
Ženski & eine & einer & einer & eine \\
Srednji & ein & eines & einem & ein \\
\bottomrule
\end{tabular}
\end{table}

\subsection{Najčešći završetci}
\begin{description}
  \item[Muški rod:]
    \nastavak{ismus}, \nastavak{or}, \nastavak{ling}, \nastavak{ig},
    \nastavak{ich} (uvijek); \nastavak{er}, \nastavak{el}, \nastavak{en}
    (često)
  \item[Ženski:]
    \nastavak{in}, \nastavak{ung}, \nastavak{schaft}, \nastavak{ei},
    \nastavak{t\"at}, \nastavak{heit}, \nastavak{keit}, \nastavak{ur},
    \nastavak{ie}, \nastavak{ion}, \nastavak{ik} (uvijek);
    \nastavak{e} (često)
  \item[Srednji:]
    \nastavak{chen}, \nastavak{lein}, \nastavak{um}, \nastavak{tum},
    \nastavak{ment}, \nastavak{ma} (uvijek). Riječi sa prefiksom \prefiks{ge}
    su obično srednjeg roda.
\end{description}

\section{Množina}
Mali broj imenica tvori množinu dodavanjem nastavka \nastavak{s} i člana \clan{die}:

\primjer{der Park}{die Parks}

\begin{description}
  \item[Muški rod:]
    dodavanjem nastavka \nastavak{e} i člana \clan{die}:

    \primjer{der Abend}{die Abende}

    Imenice koje završavaju na \nastavak{er}, \nastavak{el}, \nastavak{en} nemaju promjene nastavka u množini, ali se dodaje umlaut iznad samoglasnika:

    \primjer{der Vater}{die V\"ater}

  \item[Ženski rod:]
    dodavanjem nastavka \nastavak{n}, ili \nastavak{en}:

    \primjer{die Frau}{die Frauen}

    Ako imenica završava na \nastavak{in}, nastavak je \nastavak{nen}:

    \primjer{die Freundin}{die Freundinnen}

  \item[Srednji rod:]
    Mnoge imenice tvore množinu zamjenom člana \clan{die} bez nastavka:

    \primjer{das Fenster}{die Fenster}

    Imenice koje završavaju na \nastavak{chen} ili \nastavak{lein} tvore množinu zamjenom člana \clan{die}:

    \primjer{das M\"adchen}{die M\"adchen}

    Neke (posebno jednosložne) tvore množinu dodavanjem umlauta i nastavka \nastavak{er}, s članom \clan{die}:

    \primjer{das Fahrrad}{die Fahrr\"ader} \\
    \primjer{das Glas}{die Gl\"aser}
\end{description}

\begin{table*}[htb]
\caption{Deklinacija imenica}
\begin{tabular}{llllll}
\toprule
  &  & Muški rod & Muški rod (n-deklinacija) & Ženski rod & Srednji rod \\
\midrule
\multirow{4}{3mm}{\begin{sideways}\parbox{15mm}{Jednina}\end{sideways}}
& Nominativ & der Mann & der Junge & die Frau & das Kind \\
& Genitiv & des Mann\bf{es} & des Junge\bf{n} & der Frau & des Kind\bf{es} \\
& Dativ & dem Mann & dem Junge\bf{n} & der Frau & dem Kind \\
& Akuzativ & den Mann & den Junge\bf{n} & die Frau & das Kind \\
\midrule
\multirow{4}{3mm}{\begin{sideways}\parbox{15mm}{Množina}\end{sideways}}
& Nominativ & die M\"anner & die Junge\bf{n} & die Frauen & die Kinder \\
& Genitiv & der M\"anner & der Junge\bf{n} & der Frauen & der Kinder \\
& Dativ & den M\"anner\bf{n} & den Junge\bf{n} & den Frauen & den Kinder\bf{n} \\
& Akuzativ & die M\"anner & die Junge\bf{n} & die Frauen & die Kinder \\
\bottomrule
\end{tabular}
\begin{tablenotes}
  \small
  n-deklinacija: Imenicama u muškom rodu koje završavaju na \nastavak{e} ili
  \nastavak{oge} dodaje se nastavak \nastavak{n}. A onima koje završavaju na
  \nastavak{and}, \nastavak{ant}, \nastavak{ent}, \nastavak{ist}, \nastavak{at}
  ili \nastavak{ad} dodaje se nastavak \nastavak{en}.
\end{tablenotes}
\end{table*}

\section{Zamjenice}
\begin{table*}[htb]
\caption{Osobne zamjenice}
\begin{tabular}{ll|ll|ll}
\toprule
  \multicolumn{2}{c}{Nominativ} & \multicolumn{2}{c}{Dativ} & \multicolumn{2}{c}{Akuzativ} \\
\midrule
  ja & ich & sebi & mir & mene & mich \\
  ti/Vi & du/Sie & tebi/Vama & dir/Ihnen & tebe/Vas & dich/Sie \\
  on/ona/ono & er/sie/es & njemu/njoj/njemu & ihm/ihr/ihm & njega/nju/to & ihn/sie/es \\
  mi & wir & nama & uns & nas & uns \\
  vi & ihr & vama & euch & vas & euch \\
  oni & sie & njima & ihnen & njih & sie \\
\bottomrule
\end{tabular}
\end{table*}

\begin{table}[H]
\caption{Posvojne zamjenice}
\begin{tabular}{ll}
\toprule
Hrvatski  &  Njemački \\
\midrule
moj & mein \\
tvoj/Vaš & dein/Ihr \\
njegov/njen/njegov & sein/ihr/sein \\
naš & unser \\
vaš & euer \\
njihov & ihr \\
\bottomrule
\end{tabular}
\end{table}

Pokazne zamjenice:
\begin{itemize}[nolistsep, label={}]
    \prijevod{dieser}{ovo}
    \prijevod{jeder}{svaki}
    \prijevod{solcher}{kao}
    \prijevod{welcher}{koji}
    \prijevod{jener}{ono}
    \prijevod{mancher}{mnogi}
\end{itemize}

\section{Pridjevi}
\begin{itemize}
  \item Ako je pridjev na kraju rečenice, nema nastavka.
  \item Ako je pridjev ispred imenice koja je u \emph{definitivnom} obliku (ima član \clan{der}/\clan{die}/\clan{das}), pridjev dobiva nastavak:
  \begin{itemize}[label={}]
    \item \nastavak{e} za jedninu
    \item \nastavak{en} za množinu
  \end{itemize}
  \item Kod \emph{neodređenih} članova, pridjev preuzima oznaku roda dodavanjem nastavaka: \nastavak{r}, \nastavak{e} ili \nastavak{s} (zbog de\emph{r}, di\emph{e}, da\emph{s}). % TODO: provjeri jesu li nastavci točni; nije li možda er e es?
  % ein guter Mann
  % eine gute Frau
  % ein gutes Kind
\end{itemize}

\subsection{Komparacija pridjeva}
\begin{itemize}
  \item Komparativ se tvori dodavanja nastavka \nastavak{er}: \\
    \primjer{sch\"on}{sch\"oner}

  \item Jednosložni pridjevi sa samoglasnicima \emph{a}, \emph{o}, \emph{u}, u komparativu dobivaju umlaut: \\
    \primjer{alt}{\"alter}

  \item Superlativ se tvori dodavanjem nastavka \nastavak{st}: \\
    \primjer{klein}{kleinst}

  \item Ako pridjev završava na \nastavak{d}, \nastavak{t}, \nastavak{s}, \nastavak{ss}, \nastavak{\ss}, ili \nastavak{z}, tad se \nastavak{e} dodaje prije superlativnog nastavka \nastavak{st}: \\
    \primjer{hei{\ss}}{hei{\ss}est}

  \item Ako pridjev završava na \nastavak{er}, ili \nastavak{el}, tad gubi \nastavak{e} u komparativu i superlativu: \\
    \primjer{sauer}{saurer}
\end{itemize}

\subsection{Posebnosti akuzativa}
Ako imenica u muškom rodu ima pridjev ispred sebe, pridjevu se dodaje nastavak \nastavak{en}:
\begin{description}
  \item[Nominativ:] der alte Mann
  \item[Akuzativ:] den alten Mann
\end{description}

\section{Glagoli}
\begin{table}[H]
\caption{Konjugacija pravilnih glagola}
\begin{tabular}{lll}
\toprule
Rod   &   Nastavak  &    Primjer \\
\midrule
ja  & \nastavak{e} & ich gehe \\
ti/Vi & \nastavak{st}/\nastavak{en} & du gehst/Sie gehen \\
on/ona/ono & \nastavak{t} &  er/sie/es geht \\
mi & \nastavak{en} & wir gehen \\
vi & \nastavak{t} & ihr geht \\
oni & \nastavak{en} & sie gehen \\
\bottomrule
\end{tabular}
\end{table}

\begin{itemize}
  \item Infinitiv ima nastavak \nastavak{n} ili \nastavak{en}.
  \item Ako korijen glagola završava na \nastavak{t} ili \nastavak{d}, \nastavak{e} se dodaje prije nastavka: \\
    \primjer{arbeiten}{du arbeitest} \\
    \primjer{arbeiten}{er/sie/es arbeitet}
\end{itemize}

\begin{table}[H]
\caption{Glagol biti}
\begin{tabular}{ll}
\toprule
Hrvatski  &  Njemački \\
\midrule
ja sam   &   ich bin \\
ti si/Vi ste  &  du bist / Sie sind \\
on/ona/ono je &  er/sie/es ist \\
mi smo  &  wir sind \\
vi ste  &  ihr seid \\
oni su  &  sie sind \\
\bottomrule
\end{tabular}
\end{table}

\begin{table}[H]
\caption{Glagol imati}
\begin{tabular}{ll}
\toprule
Hrvatski  &  Njemački \\
\midrule
ja imam & ich habe \\
ti imaš/Vi imate & du hast/Sie haben \\
on/ona/ono ima & er/sie/es hat \\
mi imamo & wir haben \\
vi imate & ihr habt \\
oni imaju & sie haben \\
\bottomrule
\end{tabular}
\end{table}

%Prefiksi:
%\begin{description}
%  \item[Nerazdvojivi:] be-, ent-, emp-, er-, ge-, ver-, zer-
%  \item[Razdvojivi:] an, auf, aus, bei, ein, her, hin, mit, nach, um, weg
%% TODO: tablica str 82 everything
%% TODO: ukloniti?
%\end{description}

\section{Negacija}
\begin{itemize}
  \item \emph{nicht} uz glagole i \emph{kein} uz imenice.
  \item \emph{kein} negira \emph{ein} (\emph{kein}, \emph{keine}, \emph{kein}).
  \item Derivati negacija:
  \begin{itemize}[nolistsep, label={}]
      \prijevod{nein}{ne}
      \prijevod{keinerlei}{nijedan}
      \prijevod{keinesfalls}{nipošto}
      \prijevod{keineswegs}{ni slučajno}
      \prijevod{nichts}{ništa}
      \prijevod{nie}{nikad}
      \prijevod{niemals}{nijednom}
      \prijevod{niemand}{nitko}
      \prijevod{nirgendwo}{nigdje}
      \prijevod{weder \ldots noch \ldots}{niti \ldots ni \ldots}
  \end{itemize}
\end{itemize}

\section{Futur}
Osnova je glagol \emph{werden}.
Osim futura, \emph{werden} može značiti ``\emph{postati}:'' \\
\primjer{Ich werde alt}{Ja postajem star / Ja starim} \\
\primjer{Ich werde essen.}{Ja ću jesti.} \\
\primjer{es wird kalt.}{Postaje hladno.}

\begin{table}[H]
\begin{tabular}{ll}
\toprule
Hrvatski  &  Njemački \\
\midrule
ja ću <glagol> & ich werde <glagol u infinitivu> \\
ti ćeš/Vi ćete & du wirst/Sie werden \\
on/ona/ono će & er/sie/es wird \\
mi ćemo & wir werden \\
vi ćete & ihr werdet \\
oni će & sie werden \\
\bottomrule
\end{tabular}
\end{table}

\section{Prošlo vrijeme}
\begin{itemize}
  \item Tvori se dodavanjem nastavka \nastavak{te} na korijen riječi (kod pravilnih glagola): \\
  \primjer{fragen}{fragte}
  \item Ako korijen završava na \nastavak{t} ili \nastavak{d}, \nastavak{e} se dodaje prije nastavka \nastavak{te}:
  \primjer{warten}{wartete}

  \item Konjugacija slijedi pravila iz prezenta (samo što korijen ima nastavak).
\end{itemize}

\subsection{Prošlo svršeno vrijeme}
Tvori se dodavanjem prefiksa \prefiks{ge} i sufiksa \nastavak{t} na korijen riječi. Koristi se uz modalni glagol \emph{haben}: \\
\primjerTri{kaufen}{kauft}{habe gekauft} \\
\primjerNul{Ich habe das Buch gekauft.}

\section{Pomoćni glagoli}
TODO everything str 188-191

\section{Upitne riječi}
\begin{itemize}[nolistsep, label={}]
    \prijevod{wo}{gdje}
    \prijevod{woher}{odkud}
    \prijevod{wie}{kako}
    \prijevod{was}{što}
    \prijevod{warum}{zašto}
    \prijevod{wem}{komu}
    \prijevod{wohin}{kamo}
    \prijevod{wer}{tko}
    \prijevod{wann}{kad}
    \prijevod{was f\"ur}{kakva}
    \prijevod{wen}{koga}
    \prijevod{wessen}{čiji, čiju}
\end{itemize}

\section{Veznici}
\begin{itemize}
  \item Standardni:
  \begin{itemize}[nolistsep, label={}]
      \prijevod{oder}{ili}
      \prijevod{und}{i}
      \prijevod{aber}{ali}
      \prijevod{denn}{jer}
  \end{itemize}

  \item Veznici kod kojih glagol mora doći na kraj:
  \begin{itemize}[nolistsep, label={}]
      \prijevod{dass}{da}
      \prijevod{wenn}{ako}
      \prijevod{weil}{zbog}
      \prijevod{als}{kad}
  \end{itemize}
  \primjerNul{Ich wusste nicht, dass du in Berlin bist.}

  \item Upitne riječi (\emph{wer}, \emph{wen}, \emph{wem}, \emph{wessen}, \emph{was}, \emph{was f\"ur}, \emph{warum}, \emph{wann}, \emph{wie}, \emph{wo}, \emph{welcher}) se mogu koristiti kao veznici: \\
  \primjerNul{Ich weiss nicht, mit wem sie spricht.} \\
  \primjerNul{Er sagte nicht, was das bedeutet.}

  \item Riječice \clan{der}, \clan{die} i \clan{das} se mogu koristiti kao relativni veznici: \\
  \primjerNul{Er findet einen Hund, der alt und krank ist.}
\end{itemize}

\section{Prilozi}
\begin{itemize}[nolistsep, label={}]
    \prijevod{an}{do}
    \prijevod{auf}{na}
    \prijevod{hinter}{iza}
    \prijevod{in}{u}
    \prijevod{neben}{pokraj}
    \prijevod{\"uber}{preko, iznad}
    \prijevod{unter}{ispod}
    \prijevod{vor}{pred}
    \prijevod{zwischen}{između}
\end{itemize}

Skraćenice:
\begin{itemize}[nolistsep, label={}]
    \prijevod{an das}{ans}
    \prijevod{an dem}{am}
    \prijevod{auf das}{aufs}
    \prijevod{in das}{ins}
    \prijevod{in dem}{im}
    \prijevod{von dem}{vom}
    \prijevod{zu dem}{zum}
    \prijevod{zu der}{zur}
\end{itemize}

\section{Prijedlozi}
\begin{itemize}
  \item Prijedlozi koji zahtjevaju \emph{genitiv}:
  \begin{itemize}[nolistsep, label={}]
      \prijevod{anstatt}{umjesto}
      \prijevod{trotz}{unatoč}
      \prijevod{w\"ahrend}{tijekom}
      \prijevod{wegen}{zbog}
      \prijevod{dank}{zahvaljujući}
  \end{itemize}

  \item Prijedlozi koji zahtjevaju \emph{dativ}:
  \begin{itemize}[nolistsep, label={}]
      \prijevod{aus}{iz, van}
      \prijevod{au{\ss}er}{usput, izuzev}
      \prijevod{nach}{nakon, prema [mjestu]}
      \prijevod{von}{od (engl.\ \emph{from, of})}
      \prijevod{zu}{do, prema}
      \prijevod{bei}{uz, pri}
      \prijevod{mit}{sa}
      \prijevod{seit}{odkad / od (engl.\ \emph{since})}
  \end{itemize}

  \item Prijedlozi koji zahtjevaju \emph{akuzativ}:
  \begin{itemize}[nolistsep, label={}]
      \prijevod{bis}{prema, do}
      \prijevod{durch}{kroz}
      \prijevod{f\"ur}{za}
      \prijevod{ohne}{bez}
      \prijevod{um}{oko, na}
      \prijevod{wider, gegen}{nasuprot, protiv}
  \end{itemize}
\end{itemize}

\section{Izgovor}
TODO
% Ä – dugo eeeee
% Ü – izmedju u – i, više iiiiii
% Ö – izmedju o-e, više eeee sa skupljenim ustima
% ß – oštro sss
% 
% ei – ai eins = ains
% eu – oi deutsch = doič
% ie – iiiiiiii vier = viiir
% h –
% nemo ha ne čita se, kada je ispred samoglasnika,
% daje mu dužinu; zehn = ceen
% s – z sechs = zechs
% ch - h acht = aht, onako grleno h
% ck - k Jacke = Jake
% chs – ks sechs = zeks
% sch – š Schule= šule
% tz – c Katze = Kace
% tsch – č tschüs(s) = či(u)s
% sp –šp Sport = Šport
% st – št Stunde = Štunde
% v – f Vater = Fater
% z - c zehn= ceen

\end{multicols}

\end{document}

% Glagol sein – konjugacija
% Präsens
% ich bin
% du bist
% er ist
% wir sind
% ihr seid
% sie sind
% Präteritum
% ich war
% du warst
% er war
% wir waren
% ihr wart
% sie waren
% Perfekt
% ich bin gewesen
% du bist gewesen
% er ist gewesen
% wir sind gewesen
% ihr seid gewesen
% sie sind gewesen
% Plusquamperfekt
% ich war gewesen
% du warst gewesen
% er war gewesen
% wir waren gewesen
% ihr wart gewesen
% sie waren gewesen
% Futur I
% ich werde sein
% du wirst sein
% er wird sein
% wir werden sein
% ihr werdet sein
% sie werden sein
% Futur II
% ich werde gewesen sein
% du wirst gewesen sein
% er wird gewesen sein
% wir werden gewesen sein
% ihr werdet gewesen sein
% sie werden gewesen sein
% Kao glagol sa svojim punim značenjem, koristi se kod:
% Predstavljanja, pozicija, mesto na kome se nalazimo, kao imeniski deo predikata
