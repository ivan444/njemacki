\documentclass[12pt,german]{article}
\usepackage{amsmath}
\usepackage{amssymb}
\usepackage[croatian,german]{babel}
\usepackage{booktabs}
\usepackage[bf]{caption}
\usepackage{enumitem}
\usepackage{float}
\usepackage[T1]{fontenc}
\usepackage[a4paper, left=0.3cm, right=0.3cm, bottom=2cm, top=0.5cm]{geometry}
\usepackage{graphicx}
\usepackage[utf8]{inputenc}
\usepackage{lmodern}
\usepackage{multicol}
\usepackage{multirow}
\usepackage{rotating}
\usepackage{titlesec}
\usepackage[flushleft]{threeparttable}
\usepackage[
    left = \glqq,%
    right = \grqq%
]{dirtytalk}

\captionsetup{labelsep=space,justification=justified,singlelinecheck=off}

% Custom section: "1 | Title"
\titleformat{\section}[block]
{\normalfont\Large\bfseries}
{\thesection\hskip 9pt$|$\hskip 9pt}
{0pt}
{}

\newcommand{\nastavak}[1]{\emph{-#1}}
\newcommand{\prefiks}[1]{\emph{#1-}}
\newcommand{\clan}[1]{\emph{#1}}
\newcommand{\primjer}[2]{\say{#1 $\longmapsto$ #2}}
\newcommand{\primjerNul}[1]{\say{#1}}
\newcommand{\primjerTri}[3]{\say{#1 $\longmapsto$ #2 $\longmapsto$ #3}}
\newcommand{\prijevod}[2]{\item #1 (\emph{#2})}

\newcommand{\ja}[1]{\renewcommand{\internoja}{ja #1}}
\newcommand{\tivi}[2]{\renewcommand{\internotivi}{ti #1/Vi #2}}
\newcommand{\ono}[1]{\renewcommand{\internoono}{on/ona/ono #1}}
\newcommand{\mi}[1]{\renewcommand{\internomi}{mi #1}}
\newcommand{\vi}[1]{\renewcommand{\internovi}{vi #1}}
\newcommand{\oni}[1]{\renewcommand{\internooni}{oni #1}}
\newcommand{\internoja}{PRAZNO!}
\newcommand{\internotivi}{PRAZNO!}
\newcommand{\internoono}{PRAZNO!}
\newcommand{\internomi}{PRAZNO!}
\newcommand{\internovi}{PRAZNO!}
\newcommand{\internooni}{PRAZNO!}

\newcommand{\ich}[1]{\renewcommand{\internoich}{ich #1}}
\newcommand{\dusie}[2]{\renewcommand{\internodusie}{du #1/Sie #2}}
\newcommand{\es}[1]{\renewcommand{\internoes}{er/sie/es #1}}
\newcommand{\wir}[1]{\renewcommand{\internowir}{wir #1}}
\newcommand{\ihr}[1]{\renewcommand{\internoihr}{ihr #1}}
\newcommand{\sie}[1]{\renewcommand{\internosie}{sie #1}}
\newcommand{\internoich}{PRAZNO!}
\newcommand{\internodusie}{PRAZNO!}
\newcommand{\internoes}{PRAZNO!}
\newcommand{\internowir}{PRAZNO!}
\newcommand{\internoihr}{PRAZNO!}
\newcommand{\internosie}{PRAZNO!}

\newcommand{\pja}[1]{\renewcommand{\internopja}{ja sam #1}}
\newcommand{\ptivi}[2]{\renewcommand{\internoptivi}{ti si #1/Vi ste #2}}
\newcommand{\pon}[1]{\renewcommand{\internopon}{on (ona/ono) je #1}}
\newcommand{\pmi}[1]{\renewcommand{\internopmi}{mi smo #1}}
\newcommand{\pvi}[1]{\renewcommand{\internopvi}{vi ste #1}}
\newcommand{\poni}[1]{\renewcommand{\internoponi}{oni su #1}}
\newcommand{\internopja}{PRAZNO!}
\newcommand{\internoptivi}{PRAZNO!}
\newcommand{\internopon}{PRAZNO!}
\newcommand{\internopmi}{PRAZNO!}
\newcommand{\internopvi}{PRAZNO!}
\newcommand{\internoponi}{PRAZNO!}

\newcommand{\pich}[1]{\renewcommand{\internopich}{ich #1}}
\newcommand{\pdusie}[2]{\renewcommand{\internopdusie}{du #1/Sie #2}}
\newcommand{\pes}[1]{\renewcommand{\internopes}{er/sie/es #1}}
\newcommand{\pwir}[1]{\renewcommand{\internopwir}{wir #1}}
\newcommand{\pihr}[1]{\renewcommand{\internopihr}{ihr #1}}
\newcommand{\psie}[1]{\renewcommand{\internopsie}{sie #1}}
\newcommand{\internopich}{PRAZNO!}
\newcommand{\internopdusie}{PRAZNO!}
\newcommand{\internopes}{PRAZNO!}
\newcommand{\internopwir}{PRAZNO!}
\newcommand{\internopihr}{PRAZNO!}
\newcommand{\internopsie}{PRAZNO!}

\newenvironment{glagoltbl}[1]
{%
  \begin{table}[H]
    \caption{Glagol #1}
}{%
    \begin{tabular}{lll}
    \toprule
      & Hrvatski  &  Njemački \\
    \midrule
    \multirow{6}{2mm}{\begin{sideways}\parbox{15mm}{Sadašnjost}\end{sideways}}
      & \internoja & \internoich \\
      & \internotivi & \internodusie \\
      & \internoono & \internoes \\
      & \internomi & \internowir \\
      & \internovi & \internoihr \\
      & \internooni & \internosie \\
    \midrule
    \multirow{6}{2mm}{\begin{sideways}\parbox{15mm}{Prošlost}\end{sideways}}
      & \internopja & \internopich \\
      & \internoptivi & \internopdusie \\
      & \internopon & \internopes \\
      & \internopmi & \internopwir \\
      & \internopvi & \internopihr \\
      & \internoponi & \internopsie \\
      \bottomrule
    \end{tabular}
  \end{table}}

\newenvironment{zamjenicetbl}[1]
{%
  \begin{table}[H]
    \caption{#1 zamjenice}
    \begin{tabular}{lll}
    \toprule
      & Hrvatski  &  Njemački \\
}{%
      \bottomrule
    \end{tabular}
  \end{table}}

\newenvironment{zamjenicepadez}[1]
{%
  \midrule
  \multirow{6}{2mm}{\begin{sideways}\parbox{20mm}{#1}\end{sideways}}
}{%
}

\author{Ivan Krišto}

\begin{document}
\title{Sažetak njemačke gramatike}
\date{}
\maketitle

% 11.
% Primjeri nisu svi ispravni. Ne da mi se pisati, pitaj me u živo :P

\begin{multicols}{2}
\section{Rodovi}

\begin{table}[H]
\caption{Određeni članovi.}
\begin{tabular}{lllll}
\toprule
Rod & Nominativ & Genitiv & Dativ & Akuzativ \\
\midrule
Muški & der & des & dem & den \\
Ženski & die & der & der & die \\
Srednji & das & des & dem & das \\
Množina & die & der & den & die \\
\bottomrule
\end{tabular}
\end{table}

\begin{table}[H]
\caption{Neodređeni članovi.}
\begin{tabular}{lllll}
\toprule
Rod & Nominativ & Genitiv & Dativ & Akuzativ \\
\midrule
Muški & ein & eines & einem & einen \\
Ženski & eine & einer & einer & eine \\
Srednji & ein & eines & einem & ein \\
\bottomrule
\end{tabular}
\end{table}

\subsection{Najčešći završetci}
\begin{description}
  \item[Muški rod:]
    \nastavak{ismus}, \nastavak{or}, \nastavak{ling}, \nastavak{ig},
    \nastavak{ich} (uvijek); \nastavak{er}, \nastavak{el}, \nastavak{en}
    (često)
  \item[Ženski:]
    \nastavak{in}, \nastavak{ung}, \nastavak{schaft}, \nastavak{ei},
    \nastavak{t\"at}, \nastavak{heit}, \nastavak{keit}, \nastavak{ur},
    \nastavak{ie}, \nastavak{ion}, \nastavak{ik} (uvijek);
    \nastavak{e} (često)
  \item[Srednji:]
    \nastavak{chen}, \nastavak{lein}, \nastavak{um}, \nastavak{tum},
    \nastavak{ment}, \nastavak{ma} (uvijek). Riječi sa prefiksom \prefiks{ge}
    su obično srednjeg roda.
\end{description}

\section{Množina}
Mali broj imenica tvori množinu dodavanjem nastavka \nastavak{s} i člana
  \clan{die}:

\primjer{der Park}{die Parks}

\begin{description}
  \item[Muški rod:]
    dodavanjem nastavka \nastavak{e} i člana \clan{die}:

    \primjer{der Abend}{die Abende}

    Imenice koje završavaju na \nastavak{er}, \nastavak{el}, \nastavak{en}
    nemaju promjene nastavka u množini, ali se dodaje umlaut iznad
    samoglasnika:

    \primjer{der Vater}{die V\"ater}

  \item[Ženski rod:]
    dodavanjem nastavka \nastavak{n}, ili \nastavak{en}:

    \primjer{die Frau}{die Frauen}

    Ako imenica završava na \nastavak{in}, nastavak je \nastavak{nen}:

    \primjer{die Freundin}{die Freundinnen}

  \item[Srednji rod:]
    Mnoge imenice tvore množinu zamjenom člana \clan{die} bez nastavka:

    \primjer{das Fenster}{die Fenster}

    Imenice koje završavaju na \nastavak{chen} ili \nastavak{lein} tvore
    množinu zamjenom člana \clan{die}:

    \primjer{das M\"adchen}{die M\"adchen}

    Neke (posebno jednosložne) tvore množinu dodavanjem umlauta i nastavka
    \nastavak{er}, s članom \clan{die}:

    \primjer{das Fahrrad}{die Fahrr\"ader} \\
    \primjer{das Glas}{die Gl\"aser}
\end{description}

\begin{table*}[htb]
\caption{Deklinacija imenica}
\begin{tabular}{llllll}
\toprule
  &  & Muški rod & Muški rod (n-deklinacija) & Ženski rod & Srednji rod \\
\midrule
\multirow{4}{3mm}{\begin{sideways}\parbox{15mm}{Jednina}\end{sideways}}
& Nominativ & der Mann & der Junge & die Frau & das Kind \\
& Genitiv & des Mann\bf{es} & des Junge\bf{n} & der Frau & des Kind\bf{es} \\
& Dativ & dem Mann & dem Junge\bf{n} & der Frau & dem Kind \\
& Akuzativ & den Mann & den Junge\bf{n} & die Frau & das Kind \\
\midrule
\multirow{4}{3mm}{\begin{sideways}\parbox{15mm}{Množina}\end{sideways}}
& Nominativ & die M\"anner & die Junge\bf{n} & die Frauen & die Kinder \\
& Genitiv & der M\"anner & der Junge\bf{n} & der Frauen & der Kinder \\
& Dativ & den M\"anner\bf{n} & den Junge\bf{n} & den Frauen & den Kinder\bf{n} \\
& Akuzativ & die M\"anner & die Junge\bf{n} & die Frauen & die Kinder \\
\bottomrule
\end{tabular}
\begin{tablenotes}
  \small
  \item n-deklinacija: Imenicama u muškom rodu koje završavaju na \nastavak{e}
    ili \nastavak{oge} dodaje se nastavak \nastavak{n}.\\
    A onima koje završavaju na \nastavak{and}, \nastavak{ant}, \nastavak{ent},
    \nastavak{ist}, \nastavak{at} ili \nastavak{ad} dodaje se nastavak
    \nastavak{en}.
\end{tablenotes}
\end{table*}

\section{Zamjenice}
\begin{zamjenicetbl}{Osobne}
  \zamjenicepadez{Nominativ}
    & ja         & ich \\
    & ti/Vi      & du/Sie \\
    & on/ona/ono & er/sie/es \\
    & mi         & wir \\
    & vi         & ihr \\
    & oni        & sie \\
  \endzamjenicepadez
  \zamjenicepadez{Genitiv}
    & mene         & meiner \\
    & tebe/Vas     & deiner/Ihrer \\
    & njega/nje/njega & seiner/ihrer/seiner \\
    & nas          & unser \\
    & vas          & euer \\
    & njih         & ihrer \\
  \endzamjenicepadez
  \zamjenicepadez{Dativ}
    & sebi               & mir \\
    & tebi/Vama          & dir/Ihnen \\
    & njemu/njoj/njemu   & ihm/ihr/ihm \\
    & nama               & uns \\
    & vama               & euch \\
    & njima              & ihnen \\
  \endzamjenicepadez
  \zamjenicepadez{Akuzativ}
    & mene         & mich       \\
    & tebe/Vas     & dich/Sie   \\
    & njega/nju/to & ihn/sie/es \\
    & nas          & uns        \\
    & vas          & euch       \\
    & njih         & sie        \\
  \endzamjenicepadez
\end{zamjenicetbl}

\begin{zamjenicetbl}{Posvojne}
  \zamjenicepadez{Muški rod}
    & moj           & mein \\
    & tvoj/Vaš      & dein/Ihr \\
    & njegov/njezin/njegov & sein/ihr/sein \\
    & naš           & unser \\
    & vaš           & euer \\
    & njihov        & ihr \\
  \endzamjenicepadez
  \zamjenicepadez{Ženski rod}
    & moja         & meine \\
    & tvoja/Vaša & deine/Ihre \\
    & njegova/njezina/njegova & seine/ihre/seine \\
    & naša & unsere \\
    & vaša & euere \\
    & njihova & ihre \\
  \endzamjenicepadez
  \zamjenicepadez{Srednji rod}
    & moje   & mein \\
    & tvoje/Vaše   & dein/Ihr \\
    & njegovo/njezino/njegovo   & sein/ihr/sein \\
    & naše   & unser \\
    & vaše   & euer \\
    & njihovo   & ihr \\
  \endzamjenicepadez
  \zamjenicepadez{Množina}
    & moji/moja/moje & meine       \\
    & tvoji/tvoja/tvoje & deine \\
    & Vaši/Vaša/Vaše & Ihre \\
    & njegovi/njegova/njegove & seine  \\
    & njezini/njezina/njezine & ihre \\
    & naši/naša/naše & unsere \\
    & vaši/vaša/vaše & euere \\
    & njihovi/njihova/njihove & ihre \\
  \endzamjenicepadez
\end{zamjenicetbl}

\begin{table*}[htb]
\caption{Deklinacija posvojnih zamjenica}
\begin{tabular}{lll}
\toprule
  & Rod/Broj & Oblik \\
  \midrule
  \multirow{4}{3mm}{\begin{sideways}\parbox{20mm}{Nominativ}\end{sideways}}
  & Muški & mein Vater \\
  & Ženski & meine Mutter \\
  & Srednji & mein Buch \\
  & Množina & meine B\"ucher \\
  \midrule
  \multirow{4}{3mm}{\begin{sideways}\parbox{15mm}{Genitiv}\end{sideways}}
  & Muški & meines Vater\bf{s} \\
  & Ženski & meiner Mutter \\
  & Srednji & meines Buch\bf{es} \\
  & Množina & meiner B\"ucher \\
  \midrule
  \multirow{4}{3mm}{\begin{sideways}\parbox{12mm}{Dativ}\end{sideways}}
  & Muški & meinem Vater \\
  & Ženski & meiner Mutter \\
  & Srednji & meinem Buch \\
  & Množina & meinen B\"ucher\bf{n} \\
  \midrule
  \multirow{4}{3mm}{\begin{sideways}\parbox{18mm}{Akuzativ}\end{sideways}}
  & Muški & meinen Vater \\
  & Ženski & meine Mutter \\
  & Srednji & mein Buch \\
  & Množina & meine B\"ucher \\
  \bottomrule
\end{tabular}
\end{table*}

\begin{table}[H]
\caption{Povratne zamjenice}
\begin{tabular}{ll}
\toprule
Hrvatski  &  Njemački \\
\midrule
ja se & ich mich \\
ti se/Vi se & du dich/Sie sich \\
on/ona/ono se & er/sie/es sich \\
mi se & wir uns \\
vi se & ihr euch \\
oni se & sie sich \\
\bottomrule
\end{tabular}
\end{table}

Pokazne zamjenice:
\begin{itemize}[nolistsep, label={}]
    \prijevod{dieser}{ovo}
    \prijevod{jedermann}{svatko}
    \prijevod{jeder}{svaki}
    \prijevod{jemand}{netko}
    \prijevod{jener}{ono}
    \prijevod{mancher}{mnogi}
    \prijevod{niemand}{nitko}
    \prijevod{solcher}{kao}
\end{itemize}

\begin{table}[H]
\caption{Pokazna zamjenica \emph{der}.}
\begin{tabular}{lllll}
\toprule
Rod & Nominativ & Genitiv & Dativ & Akuzativ \\
\midrule
Muški & der & dessen & dem & den \\
Ženski & die & deren & der & die \\
Srednji & das & dessen & dem & das \\
Množina & die & deren & denen & die \\
\bottomrule
\end{tabular}
\end{table}

\section{Pridjevi}
\begin{itemize}
  \item Ako je pridjev na kraju rečenice, nema nastavka.
  \item Ako je pridjev ispred imenice koja je u \emph{definitivnom} obliku (ima
    član \clan{der}/\clan{die}/\clan{das}), pridjev dobiva nastavak:
  \begin{itemize}[label={}]
    \item \nastavak{e} za jedninu
    \item \nastavak{en} za množinu
  \end{itemize}
  U \emph{akuzativu} nastavak za muški rod je \nastavak{en}.
  \item Kod \emph{neodređenih} članova, pridjev preuzima oznaku roda dodavanjem
    nastavaka: \nastavak{er}, \nastavak{e} ili \nastavak{es} (zbog der, die,
    das). Nastavci u akuzativu: \nastavak{en}, \nastavak{es}, \nastavak{e}.
\end{itemize}

\subsection{Komparacija pridjeva}
\begin{itemize}
  \item Komparativ se tvori dodavanja nastavka \nastavak{er}: \\
    \primjer{sch\"on}{sch\"oner}

  \item Jednosložni pridjevi sa samoglasnicima \emph{a}, \emph{o}, \emph{u}, u
    komparativu dobivaju umlaut: \\
    \primjer{alt}{\"alter}

  \item Superlativ se tvori se pomoću \emph{am} i nastavka \nastavak{sten} ili
    \nastavak{esten}: \\
    \primjer{klein}{am kleinsten}

  \item Ako pridjev završava na \nastavak{d}, \nastavak{t}, \nastavak{s},
    \nastavak{ss}, \nastavak{\ss}, ili \nastavak{z}, tad se \nastavak{e} dodaje
    prije superlativnog nastavka \nastavak{st}: \\
    \primjer{hei{\ss}}{hei{\ss}est}

  \item Ako pridjev završava na \nastavak{er}, ili \nastavak{el}, tad gubi
    \nastavak{e} u komparativu i superlativu: \\
    \primjer{sauer}{saurer}
\end{itemize}

\section{Glagoli}
\begin{table}[H]
\caption{Konjugacija pravilnih glagola}
\begin{tabular}{lll}
\toprule
Rod   &   Nastavak  &    Primjer \\
\midrule
ja  & \nastavak{e} & ich gehe \\
ti/Vi & \nastavak{st}/\nastavak{en} & du gehst/Sie gehen \\
on/ona/ono & \nastavak{t} &  er/sie/es geht \\
mi & \nastavak{en} & wir gehen \\
vi & \nastavak{t} & ihr geht \\
oni & \nastavak{en} & sie gehen \\
\bottomrule
\end{tabular}
\end{table}

\begin{itemize}
  \item Infinitiv ima nastavak \nastavak{n} ili \nastavak{en}.
  \item Ako korijen glagola završava na \nastavak{t} ili \nastavak{d},
    \nastavak{e} se dodaje prije nastavka: \\
    \primjer{arbeiten}{du arbeitest} \\
    \primjer{arbeiten}{er/sie/es arbeitet}
\end{itemize}

\begin{glagoltbl}{biti}
  \ja{sam}  \ich{bin}
  \tivi{si}{ste} \dusie{bist}{sind}
  \ono{je}  \es{ist}
  \mi{smo} \wir{sind}
  \vi{ste} \ihr{seid}
  \oni{su}  \sie{sind}

  \pja{bio}  \pich{war}
  \ptivi{bio}{bili} \pdusie{warst}{waren}
  \pon{bio}  \pes{war}
  \pmi{bili} \pwir{waren}
  \pvi{bili} \pihr{wart}
  \poni{bili}  \psie{waren}
\end{glagoltbl}

\begin{glagoltbl}{imati}
  \ja{imam}  \ich{habe}
  \tivi{imaš}{imate} \dusie{hast}{haben}
  \ono{ima}  \es{hat}
  \mi{imamo} \wir{haben}
  \vi{imate} \ihr{habt}
  \oni{imaju}  \sie{haben}

  \pja{imao}  \pich{hatte}
  \ptivi{imao}{imali} \pdusie{hattest}{hatten}
  \pon{imao}  \pes{hatte}
  \pmi{imali} \pwir{hatten}
  \pvi{imali} \pihr{hattet}
  \poni{imali}  \psie{hatten}
\end{glagoltbl}

\begin{glagoltbl}{smjeti}
  \ja{smijem}  \ich{darf}
  \tivi{smiješ}{smijete} \dusie{darfst}{d\"urfen}
  \ono{smije}  \es{darf}
  \mi{smijemo} \wir{d\"urfen}
  \vi{smijete} \ihr{d\"urft}
  \oni{smiju}  \sie{d\"urfen}

  \pja{smio}  \pich{durfte}
  \ptivi{smio}{smjeli} \pdusie{durftest}{durften}
  \pon{smio}  \pes{durfte}
  \pmi{smjeli} \pwir{durften}
  \pvi{smjeli} \pihr{durftet}
  \poni{smjeli}  \psie{durften}
\end{glagoltbl}

\begin{glagoltbl}{moći}
  \ja{mogu}  \ich{kann}
  \tivi{možeš}{možete} \dusie{kannst}{k\"onnen}
  \ono{može}  \es{kann}
  \mi{možemo} \wir{k\"onnen}
  \vi{možete} \ihr{k\"onnt}
  \oni{mogu}  \sie{k\"onnen}

  \pja{mogao}  \pich{konnte}
  \ptivi{mogao}{mogli} \pdusie{konntest}{konnten}
  \pon{mogao}  \pes{konnte}
  \pmi{mogli} \pwir{konnten}
  \pvi{mogli} \pihr{konntet}
  \poni{mogli}  \psie{konnten}
\end{glagoltbl}

\begin{glagoltbl}{morati}
  \ja{moram}  \ich{muss}
  \tivi{moraš}{morate} \dusie{musst}{m\"ussen}
  \ono{mora}  \es{muss}
  \mi{moramo} \wir{m\"ussen}
  \vi{morate} \ihr{m\"usst}
  \oni{moraju}  \sie{m\"ussen}

  \pja{morao}  \pich{musste}
  \ptivi{morao}{morali} \pdusie{musstest}{mussten}
  \pon{morao}  \pes{musste}
  \pmi{morali} \pwir{mussten}
  \pvi{morali} \pihr{musstet}
  \poni{morali}  \psie{mussten}
\end{glagoltbl}

\begin{glagoltbl}{trebati}
  \ja{trebam}  \ich{soll}
  \tivi{trebaš}{trebate} \dusie{sollst}{sollen}
  \ono{treba}  \es{soll}
  \mi{trebamo} \wir{sollen}
  \vi{trebate} \ihr{sollt}
  \oni{trebaju}  \sie{sollen}

  \pja{trebao}  \pich{sollte}
  \ptivi{trebao}{trebali} \pdusie{solltest}{sollten}
  \pon{trebao}  \pes{sollte}
  \pmi{trebali} \pwir{sollten}
  \pvi{trebali} \pihr{solltet}
  \poni{trebali}  \psie{sollten}
\end{glagoltbl}

\begin{glagoltbl}{htjeti}
  \ja{hoću}  \ich{will}
  \tivi{hoćeš}{hoćete} \dusie{willst}{wollen}
  \ono{hoće}  \es{will}
  \mi{hoćemo} \wir{wollen}
  \vi{hoćete} \ihr{wollt}
  \oni{hoće}  \sie{wollen}

  \pja{htio}  \pich{wollte}
  \ptivi{htio}{htjeli} \pdusie{wolltest}{wollten}
  \pon{htio}  \pes{wollte}
  \pmi{htjeli} \pwir{wollten}
  \pvi{htjeli} \pihr{wolltet}
  \poni{htjeli}  \psie{wollten}
\end{glagoltbl}

\begin{glagoltbl}{željeti}
  \ja{želim}  \ich{m\"ochte}
  \tivi{želiš}{želite} \dusie{m\"ochtest}{m\"ochten}
  \ono{želi}  \es{m\"ochte}
  \mi{želimo} \wir{m\"ochten}
  \vi{želite} \ihr{m\"ochtet}
  \oni{žele}  \sie{m\"ochten}

  \pja{želim}  \pich{mochte}
  \ptivi{želiš}{želite} \pdusie{mochtest}{mochten}
  \pon{želi}  \pes{mochte}
  \pmi{želimo} \pwir{mochten}
  \pvi{želite} \pihr{mochtet}
  \poni{žele}  \psie{mochten}
\end{glagoltbl}

\subsection{Prošlo vrijeme}
\begin{itemize}
  \item Tvori se dodavanjem nastavka \nastavak{te} na korijen riječi (kod
    pravilnih glagola): \\
  \primjer{fragen}{fragte}
  \item Ako korijen završava na \nastavak{t} ili \nastavak{d}, \nastavak{e} se
    dodaje prije nastavka \nastavak{te}:
  \primjer{warten}{wartete}

  \item Konjugacija slijedi pravila iz prezenta (samo što korijen ima nastavak).
\end{itemize}

\subsubsection{Prošlo svršeno vrijeme}
Tvori se dodavanjem prefiksa \prefiks{ge} i sufiksa \nastavak{t} na korijen
riječi. Koristi se uz pomoćni glagol \emph{haben}: \\
\primjerTri{kaufen}{kauft}{habe gekauft} \\
\primjerNul{Ich habe das Buch gekauft.} \\ \newline
Nepravilni glagoli najčešće imaju sufiks \nastavak{en}: \\
\primjerNul{Ich habe gesehen.} \\ \newline
Glagoli koji izražavaju \emph{kretanje} tvore prošlo svršeno vrijeme pomoćnim
glagolom \emph{sein}: \\
\primjerNul{Ich bin gefahren.}

\subsection{Futur}
Tvori se pomoću \emph{werden} i glagola u infinitivu.  Osim futura,
\emph{werden} može značiti ``\emph{postati}:'' \\
\primjer{Ich werde alt}{Ja postajem star / Ja starim} \\
\primjer{Ich werde essen.}{Ja ću jesti.} \\
\primjer{es wird kalt.}{Postaje hladno.}

\begin{glagoltbl}{\emph{werden} (\glqq ja ću\grqq\ ili \glqq postati\grqq)}
  \ja{ću}  \ich{werde}
  \tivi{ćeš}{ćete} \dusie{wirst}{werden}
  \ono{će}  \es{wird}
  \mi{ćemo} \wir{werden}
  \vi{ćete} \ihr{werdet}
  \oni{će}  \sie{werden}

  \pja{postao}  \pich{wurde}
  \ptivi{postao}{postali} \pdusie{wurdest}{wurden}
  \pon{postao}  \pes{wurde}
  \pmi{postali} \pwir{wurden}
  \pvi{postali} \pihr{wurdet}
  \poni{postali}  \psie{wurden}
\end{glagoltbl}

\section{Negacija}
\begin{itemize}
  \item \emph{nicht} uz glagole i \emph{kein} uz imenice.
  \item \emph{kein} negira \emph{ein} (\emph{kein}, \emph{keine}, \emph{kein}).
  \item Derivati negacija:
  \begin{itemize}[nolistsep, label={}]
      \prijevod{nein}{ne}
      \prijevod{keinerlei}{nijedan}
      \prijevod{keinesfalls}{nipošto}
      \prijevod{keineswegs}{ni slučajno}
      \prijevod{nichts}{ništa}
      \prijevod{nie}{nikad}
      \prijevod{niemals}{nijednom}
      \prijevod{niemand}{nitko}
      \prijevod{nirgendwo}{nigdje}
      \prijevod{weder \ldots noch \ldots}{niti \ldots ni \ldots}
  \end{itemize}
\end{itemize}

\section{Upitne riječi}
\begin{itemize}[nolistsep, label={}]
    \prijevod{wann}{kad}
    \prijevod{warum}{zašto}
    \prijevod{was f\"ur}{kakva}
    \prijevod{was}{što}
    \prijevod{welcher}{koji}
    \prijevod{wem}{komu}
    \prijevod{wen}{koga}
    \prijevod{wer}{tko}
    \prijevod{wessen}{čiji, čiju}
    \prijevod{wie}{kako}
    \prijevod{wo}{gdje}
    \prijevod{woher}{odkud}
    \prijevod{wohin}{kamo}
\end{itemize}

\section{Veznici}
\begin{itemize}
  \item Standardni:
  \begin{itemize}[nolistsep, label={}]
      \prijevod{aber}{ali}
      \prijevod{denn}{jer}
      \prijevod{oder}{ili}
      \prijevod{und}{i}
  \end{itemize}

  \item Veznici kod kojih glagol mora doći na kraj:
  \begin{itemize}[nolistsep, label={}]
      \prijevod{als}{kad}
      \prijevod{dass}{da}
      \prijevod{weil}{zbog}
      \prijevod{wenn}{ako}
  \end{itemize}
  \primjerNul{Ich wusste nicht, dass du in Berlin bist.}

  \item Upitne riječi (\emph{wer}, \emph{wen}, \emph{wem}, \emph{wessen},
    \emph{was}, \emph{was f\"ur}, \emph{warum}, \emph{wann}, \emph{wie},
    \emph{wo}, \emph{welcher}) se mogu koristiti kao veznici: \\
  \primjerNul{Ich weiss nicht, mit wem sie spricht.} \\
  \primjerNul{Er sagte nicht, was das bedeutet.}

  \item Riječice \clan{der}, \clan{die} i \clan{das} se mogu koristiti kao
    relativni veznici: \\
  \primjerNul{Er findet einen Hund, der alt und krank ist.}
\end{itemize}

\section{Prilozi}
\begin{itemize}[nolistsep, label={}]
    \prijevod{an}{do}
    \prijevod{auf}{na}
    \prijevod{hinter}{iza}
    \prijevod{in}{u}
    \prijevod{neben}{pokraj}
    \prijevod{\"uber}{preko, iznad}
    \prijevod{unter}{ispod}
    \prijevod{vor}{pred}
    \prijevod{zwischen}{između}
\end{itemize}

Skraćenice:
\begin{itemize}[nolistsep, label={}]
    \prijevod{an das}{ans}
    \prijevod{an dem}{am}
    \prijevod{auf das}{aufs}
    \prijevod{in das}{ins}
    \prijevod{in dem}{im}
    \prijevod{von dem}{vom}
    \prijevod{zu dem}{zum}
    \prijevod{zu der}{zur}
\end{itemize}

\section{Prijedlozi}
\begin{itemize}
  \item Prijedlozi koji zahtjevaju \emph{genitiv}:
  \begin{itemize}[nolistsep, label={}]
      \prijevod{anstatt}{umjesto}
      \prijevod{au{\ss}erhalb}{izvan}
      \prijevod{dank}{zahvaljujući}
      \prijevod{infolge}{zbog, uslijed}
      \prijevod{innerhalb}{unutar}
      \prijevod{jenseits}{s one strane}
      \prijevod{statt}{umjesto}
      \prijevod{trotz}{unatoč}
      \prijevod{w\"ahrend}{tijekom, dok}
      \prijevod{wegen}{zbog}
  \end{itemize}

  \item Prijedlozi koji zahtjevaju \emph{dativ}:
  \begin{itemize}[nolistsep, label={}]
      \prijevod{aus}{iz, van}
      \prijevod{au{\ss}er}{usput, osim, izuzev}
      \prijevod{bei}{uz, kod, pri}
      \prijevod{entgegen}{protiv, u susret}
      \prijevod{gegen\"uber}{nasuprot, prema}
      \prijevod{mit}{sa}
      \prijevod{nach}{iza, poslije, nakon, prema [mjestu]}
      \prijevod{seit}{odkad / od (engl.\ \emph{since})}
      \prijevod{von}{od (engl.\ \emph{from, of})}
      \prijevod{zu}{do, prema, u}
  \end{itemize}

  \item Prijedlozi koji zahtjevaju \emph{akuzativ}:
  \begin{itemize}[nolistsep, label={}]
      \prijevod{bis}{prema, do, dok}
      \prijevod{durch}{kroz, preko}
      \prijevod{entlang}{uzduž, pokraj, pored}
      \prijevod{f\"ur}{za, umjesto}
      \prijevod{ohne}{bez}
      \prijevod{um}{oko, na, u}
      \prijevod{wider, gegen}{nasuprot, protiv, prema}
  \end{itemize}

  \item Prijedlozi koji zahtjevaju \emph{dativ} ili \emph{akuzativ}:
  \begin{itemize}[nolistsep, label={}]
      \prijevod{an}{na, o, od, do, po, za, uz, u (vremenski)}
      \prijevod{auf}{na, po, za}
      \prijevod{hinter}{iza, za}
      \prijevod{in}{u, na, tokom}
      \prijevod{neben}{pokraj, osim}
      \prijevod{\"uber}{iznad, preko}
      \prijevod{unter}{ispod}
      \prijevod{vor}{ispred, prije}
      \prijevod{zwischen}{između}
  \end{itemize}
\end{itemize}

% \section{Izgovor}
% TODO
% Ä – dugo eeeee
% Ü – izmedju u – i, više iiiiii
% Ö – izmedju o-e, više eeee sa skupljenim ustima
% ß – oštro sss
% 
% ei – ai eins = ains
% eu – oi deutsch = doič
% ie – iiiiiiii vier = viiir
% h –
% nemo ha ne čita se, kada je ispred samoglasnika,
% daje mu dužinu; zehn = ceen
% s – z sechs = zechs
% ch - h acht = aht, onako grleno h
% ck - k Jacke = Jake
% chs – ks sechs = zeks
% sch – š Schule= šule
% tz – c Katze = Kace
% tsch – č tschüs(s) = či(u)s
% sp –šp Sport = Šport
% st – št Stunde = Štunde
% v – f Vater = Fater
% z - c zehn= ceen

\end{multicols}

\end{document}

% Glagol sein – konjugacija
% Futur I
% ich werde sein
% du wirst sein
% er wird sein
% wir werden sein
% ihr werdet sein
% sie werden sein
% Futur II
% ich werde gewesen sein
% du wirst gewesen sein
% er wird gewesen sein
% wir werden gewesen sein
% ihr werdet gewesen sein
% sie werden gewesen sein
% Kao glagol sa svojim punim značenjem, koristi se kod:
% Predstavljanja, pozicija, mesto na kome se nalazimo, kao imeniski deo predikata

% TODO: pasiv, str 21-23 sa file:///media/laputa/Downloads/Gramatika%20njema%C4%8Dkog%20jezika.pdf
